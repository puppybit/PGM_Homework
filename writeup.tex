\documentclass[12pt,a4paper]{article}
\usepackage{amsmath,amssymb,fullpage,graphicx}
\let\hat\widehat
\let\tilde\widetilde

\usepackage{amssymb}
\usepackage{amsmath}
\usepackage{epsfig, graphics}
\usepackage{latexsym}
\usepackage{fullpage}
\usepackage{bm}
\usepackage[parfill]{parskip}
\usepackage{subfigure}
% TODO : Fil your personal package here if needed

%%%% new version of enumerate with less spacing
\newenvironment{enum}{
\begin{enumerate}
  \setlength{\itemsep}{1pt}
  \setlength{\parskip}{0pt}
  \setlength{\parsep}{0pt}
}{\end{enumerate}}

% TODO : fill your custom commands here if needed
\newcommand{\Ab}{\bm{A}}

\begin{document}

% TODO : Fill your personal information here
\begin{center}
    {\bf\large Homework 1} \\
    {\bf\large M1522.001300 Probabilistic Graphical Models (2017 Fall)} \\
    2014-31117 Hyun Gi Ahn \\
    % Due: September 22 Thursday 11:59PM \\
    Date: September 22 Thursday
\end{center}

% TODO : Fill your answers
%%%%%%%%%%%%%%%%%%%%%%%%%%%%%%%%%%%%%%%%%%%%%%
% Q1
%%%%%%%%%%%%%%%%%%%%%%%%%%%%%%%%%%%%%%%%%%%%%%
\section{Change of Random Variables [5 points]}

\text{If} $a > 0$, \text{then cdf} ${F}_{Y}(y) = P\left[x\leq\frac{Y-b}{a}\right] = F_{X}\left(\frac{Y-b}{a}\right)$ \\
\\
\text{On the other hand, if} $a < 0$, \text{then cdf} ${F}_{Y}(y) = P\left[x\geq\frac{Y-b}{a}\right] = 1-F_{X}\left(\frac{Y-b}{a}\right)$ \\
\\
We can get the pdf of $Y$ by differentiating with repect to $y$ \\
\\
If $a>1$, then ${f}_{Y}(y)=\frac{1}{a}{f}_{x}\left(\frac{y-b}{a}\right)$ and
If $a<1$, then ${f}_{Y}(y)=-\frac{1}{a}{f}_{x}\left(\frac{y-b}{a}\right)$ \\
As a result, $${f}_{Y}(y)={\frac{1}{\left|a\right|}}{f}_{x}\left(\frac{y-b}{a}\right)\xrightarrow{}(1)$$
Gaussian pdf with mean $m$ and standard deviation $\sigma$
$${f}_{X}(x)=\frac{1}{\sqrt{2\pi\left|a\sigma\right|}}{e}^{\frac{-(x-m)^2}{2\sigma^2}}\xrightarrow{}(2)$$
Substitution of Eq. (2) into Eq. (1) yields
$${f}_{Y}(y)=\frac{1}{\sqrt{2\pi\left|a\sigma\right|}}{e}^{\frac{-(y-b-am)^2}{2a\sigma^2}}$$
Therfore $N(am+b;a^2\sigma^2)$
\section{Conditional and Total Probability [10 points]}

\begin{enumerate}
    \item $$P(Department\;A)=0.62\times\frac{825}{825+108}+0.82\times\frac{108}{825+108}=0.64$$ \\
          $$P(Department\;C)=0.33\times\frac{417}{417+375}+0.35\times\frac{375}{417+375}=0.33$$ \\

    \item $$P(Men)=\frac{0.62\times825+0.63\times560+0.33\times417+0.06\times272}{825+108+560+25+417+375+272+341}=0.34$$
          $$P(Women)=\frac{0.82\times108+0.68\times25+0.35\times375+0.07\times341}{825+108+560+25+417+375+272+341}=0.08$$

    \item Because When Department C and D is given, the probability of Woman is low. In case of Woman Applicants of Department C, D is Higher than A, B. It means that the number of admitted students in Department C and D can affect to admission rate a lot
\end{enumerate}

\section{Independence Properties [20 points]}

\begin{enumerate}
\item $(X,W\perp Y\mid Z) \xrightarrow{symmetry}(Y\perp X,W\mid Z) \xrightarrow{Decomposition}(Y\perp X\mid Z)\;\text{and}\;(Y\perp W\mid Z)
\xrightarrow{symmetry}\mathbf{(X\perp Y\mid Z)}\;\text{and}\;(W\perp Y\mid Z)$ \\

\item $(X,W\perp Y\mid Z) \xrightarrow{symmetry}(Y\perp X,W\mid Z) \xrightarrow{Weak Union}(Y\perp X\mid Z,W)\;\text{and}\;(Y\perp W\mid Z)
\xrightarrow{symmetry}\mathbf{(X\perp Y\mid W,Z)}\;\text{and}\;(W\perp Y\mid Z)$ \\

\item

\item $(X\perp Y\mid Z,W)\;\text{and}\;(Y\perp W\mid Z,X)\xrightarrow{symmetry}(Y\perp X\mid Z,W)\;\text{and}\;(Y\perp W\mid X,Z)\xrightarrow{Intersection}(Y\perp X,W\mid Z)\xrightarrow{symmetry}\mathbf{(X,W\perp Y\mid Z)}$
\end{enumerate}

\section{Naive Bayes Example [15 points]}

\begin{enumerate}

\item By Chain Rules
\begin{align}
  \nonumber P(C,X_1,...,X_n)&=P(C)P(X_1,...,X_n\mid C)\\
  \nonumber &=P(C)P(X_1\mid C)P(X_2,...,X_n\mid C,X_1)\\
  \nonumber &=P(C)P(X_1\mid C)P(X_2\mid C,X_1)P(X_3,...,X_n\mid C,X_1,X_2)\\
  \nonumber &=P(C)P(X_1\mid C)P(X_2\mid C,X_1)\;...\;P(X_n\mid C,X_1,X_2,\;...\;X_{n-1})
\end{align}
By Conditional independence property of Naive Bayes $i\neq j,k,l$
\begin{align}
  \nonumber P(x_i\mid C,x_j)=P(x_i\mid C),\\
  \nonumber P(x_i\mid C,x_j,x_k)=P(x_i\mid C),\\
  \nonumber P(x_i\mid C,x_j,x_k,x_l)=P(x_i\mid C),
\end{align}
So we can get following result
\begin{align}
  \nonumber P(C,X_1,...,X_n)&= P(C)P(X_1,...,X_n\mid C)\\
  \nonumber &= P(C)P(x_1\mid C)P(x_3\mid C)P(x_2\mid C)\;...\\
  \nonumber &= P(C)\prod_{i=1}^{n}P(x_i\mid C)\\
\end{align}
\item
\begin{align}
  \nonumber \frac{P(C=1\mid x_1,...,x_n)}{P(C=0\mid x_1,...,x_n)} &= \frac{\frac{P(C=1,x_1,...,x_n)}{P(x_1,...,x_n)}}{\frac{P(C=0,x_1,...,x_n)}{P(x_1,...,x_n)}}\\
  \nonumber &=\frac{P(C=1,x_1,...,x_n)}{P(C=0,x_1,...,x_n)}\\
  \nonumber &=\frac{P(C=1)}{P(C=0)}\frac{\prod_{i=1}^{n}P(x_i\mid C=1)}{\prod_{i=1}^{n}P(x_i\mid C=0)}=\frac{P(C=1)}{P(C=0)}\prod_{i=1}^{n}\frac{P(x_i\mid C=1)}{P(x_i\mid C=0)}
\end{align}
\item
\begin{align}
  \nonumber \frac{P(C=1\mid x_1,...,x_n)}{P(C=0\mid x_1,...,x_n)}&=\frac{P(C=1)}{P(C=0)}\prod_{i=1}^{n}\frac{P(x_i\mid C=1)}{P(x_i\mid C=0)}\\
  \nonumber &=\sum_{i=1}^{n}\log\frac{P(x_i\mid C=1)}{P(x_i\mid C=0)}+\log{P(C=1)}-\log{P(C=0)}
\end{align}
\end{enumerate}

\section{Graphical Model and Independence [10 points]}

\begin{enumerate}
  \item TRUE
  \item TRUE
  \item FALSE
  \item TRUE
  \item FALSE
  \item FALSE
  \item FALSE
  \item TRUE
  \item TRUE
  \item TRUE
\end{enumerate}

\section{V-structure [15 points]}
\begin{enumerate}
  \item
  \begin{align}
  \nonumber P(C_1=H)=\frac{1}{2},\quad P(C_1=T)=\frac{1}{2}\\
  \nonumber P(C_2=H)=\frac{1}{2},\quad P(C_2=T)=\frac{1}{2}\\
  \end{align}
  $P(C_1=H,C_2=H)=\frac{1}{4}=P(C_1=H)P(C_2=H)=\frac{1}{2}\times\frac{1}{2}$\\
  $P(C_1=H,C_2=T)=\frac{1}{4}=P(C_1=H)P(C_2=T)=\frac{1}{2}\times\frac{1}{2}$\\
  $P(C_1=T,C_2=H)=\frac{1}{4}=P(C_1=T)P(C_2=H)=\frac{1}{2}\times\frac{1}{2}$\\
  $P(C_1=T,C_2=T)=\frac{1}{4}=P(C_1=T)P(C_2=T)=\frac{1}{2}\times\frac{1}{2}$\\
  Therefore, $P(C_1)\perp P(C_2)$
  \item
  $$P(C_2=H|B=F)=\frac{P(B=F|C_2=H)P(C_2=H)}{P(B=F)}=\frac{0.5\times0.5}{0.75}=\frac{1}{3}$$
  \item
  $$P(C_2=H\mid B=F,C_1=T)=\frac{1}{2}$$
  $$P(C_1=H,C_2=H\mid B=T)=1\quad P(C_1=H\mid B=T)P(C_2=H\mid B=T)=1\cdot1=1$$
  $$P(C_1=H,C_2=T\mid B=F)=\frac{1}{3}\quad P(C_1=H\mid B=F)P(C_2=T\mid B=F)=1\cdot\frac{1}{3}=\frac{1}{3}$$
  Therefore, $P(C_1,C_2\mid B)=P(C_1\mid B)P(C_2\mid B)$\\
  $C_1\mid B$ and $C_2\mid B$ are independent

\section{I-Equivalence [10 points]}


\section{Factorization Theorem [15 points]}


\end{enumerate}



% % section conditional_and_total_probability_10_points_ (end)

% You can insert your figure by using \verb|\begin{figure}|.
% You can refer your figure by using \verb|\ref{fig:example_figure}|.

% \begin{figure}[!h]
%     \begin{center}
%         \includegraphics[width=0.7\textwidth]{assets/fig1.png}
%         \caption{Example Figure}
%         \label{fig:example_figure}
%     \end{center}
% \end{figure}

% You can enumerate sub questions like this.

% \begin{enumerate}
%     \item Enumerate item 1.
%     \item Enumerate item 2.
%     \item Enumerate item 3.
% \end{enumerate}

% You can write your equations by using \verb|\begin{aligned}|.
% We highly recommend you to use \verb|\newcommand| to simplify your equation in \LaTeX.

% \begin{align}
%     \Ab &= 12 \\
%     \alpha_{12}^{35} &= 1234 \\
%     \beta_1 &= 10
% \end{align}

% You can cite your reference by using \verb|\cite{reference_name}|.
% For example, cite Koller's PGM book \cite{KollerPGM} like this.


% %%%%%%%%%%%%%%%%%%%%%%%%%%%%%%%%%%%%%%%%%%%%%%
% % Q2
% %%%%%%%%%%%%%%%%%%%%%%%%%%%%%%%%%%%%%%%%%%%%%%
% \section{Some materials for HW2 [5 points]}

% \begin{table}[!ht]
% \centering
% \begin{tabular}{c|l|l|l|l}
% Step & \multicolumn{1}{c|}{\begin{tabular}[c]{@{}c@{}}Variable\\ eliminated\end{tabular}} & \multicolumn{1}{c|}{\begin{tabular}[c]{@{}c@{}}Factors\\ used\end{tabular}} & \multicolumn{1}{c|}{\begin{tabular}[c]{@{}c@{}}Variables\\ involved\end{tabular}} & \multicolumn{1}{c}{\begin{tabular}[c]{@{}c@{}}New\\ factor\end{tabular}} \\ \hline
% 1    &                                                                                    &                                                                             &                                                                                   &                                                                          \\
% 2    &                                                                                    &                                                                             &                                                                                   &                                                                          \\
% 3    &                                                                                    &                                                                             &                                                                                   &                                                                          \\
% 4    &                                                                                    &                                                                             &                                                                                   &                                                                          \\
% 5    &                                                                                    &                                                                             &                                                                                   &                                                                          \\
% 6    &                                                                                    &                                                                             &                                                                                   &                                                                          \\
% 7    &                                                                                    &                                                                             &                                                                                   &                                                                          \\
% 8    &                                                                                    &                                                                             &                                                                                   &
% \end{tabular}
% \caption{A run of variable elimination for the query $P(J)$}
% \label{table:VE}
% \end{table}

% Bibtex
\bibliography{writeup}
\bibliographystyle{plain}

\end{document}
